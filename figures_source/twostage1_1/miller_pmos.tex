\documentclass[]{standalone}
\usepackage{tikz}
\usepackage[siunitx]{circuitikz} % Loading circuitikz with siunitx option
%\usetikzlibrary{positioning}
\usetikzlibrary{calc}

\newcommand*{\xMin}{0}%
\newcommand*{\xMax}{4}%
\newcommand*{\yMin}{0}%
\newcommand*{\yMax}{-4}%

\begin{document}
	\begin{circuitikz}
		\ctikzset{tripoles/mos style=arrows}
		\ctikzset{transistors/arrow pos=end}
		\coordinate (start) at (0,0);
		
		%cm
		\draw (start) node[pmos, anchor=source, xscale=-1](m1){};
		\draw (m1.base) node[left=20]{$M2$};
%		\draw (m1.gate) node[circ]{} |- (m1.drain) node[circ]{};
		\draw (m1.gate) node[circ]{} |- ($(m1.drain)+(0,.2)$) node[circ]{};
		\draw (start) ++(3,0) node[pmos, anchor=source](m2){};
%		\draw (m2.base) node[below right=150]{$M2$};
		\draw (m2.drain) node[above right]{$M8$};
		\draw (m1.gate) -- (m2.gate) node(con1){}; %netBias
		\draw (m1.source) -- (m2.source); %netVdd
		\draw (m1.D) +(0,-.4) to[short, o-, i=$I_{b}$] (m1.D); %Bias current in
%		\path ($(m1.gate)!0.5!(m2.gate)$) node[below]{$net_{bias}$};
		
		
		%VDD
		\draw (m2.S) to[short,*-o] ++(0,0.5) node[right]{$V_{dd}$};
%		\path ($(m1.source)!0.5!(m2.source)$) node[above]{$net_{v_{dd}}$};
		
		%dp
		\draw (m2.drain) ++(-1.2,-.2) node[pmos, anchor=source](m3){};
		\draw (m3.base) node[right=20]{$M7$};
		\draw (m2.drain) ++(+1.2,-.2) node[pmos, anchor=source, xscale=-1](m4){};
		\draw (m4.base) node[left=20]{$M6$};
		\draw (m3.source) -- (m4.source);
		\draw (m2.drain) to[short, -*] ++(0,-.2); %net1
%		\path ($(m3.source)!0.5!(m4.source)$) node[below]{$net_{1}$};
		
		%input connections
		\draw (m3.G) ++(0,0) to[short, o-] (m3.G);
		\draw (m3.G) ++(0,-.8) to[short, o-] ++(3.5, 0) -| ($(m4.G)+(0,0)$);
%		\draw (m3.G) ++(-1.2, -.75)  node[]{$V_{in}$};
		\draw (m3.G) ++(-.1, -.2)  node[]{$V_{in-}$};
		\draw (m3.G) ++(-.1, -1)  node[]{$V_{in+}$};
		
		%cml
		\path (0,-.4) coordinate(lower-transistors-offset);
		\draw (m3.drain) ++(lower-transistors-offset) node[nmos, anchor=drain, xscale=-1](m5){};
		\draw (m5.base) node[left=30]{$M1$};
		\draw (m3.drain) -- (m5.drain);
		\draw (m4.drain) ++(lower-transistors-offset) node[nmos, anchor=drain](m6){};
		\draw ($(m5.gate)+(-.16,0)$) node[circ]{} |- ($(m5.drain)+(0,-.2)$) node[circ]{};%net2
%		\path ($(m5.drain)!0.2!(m3.source)$) node[right]{$net_{2}$};
		\draw (m6.base) node[right=30]{$M4$};
		\draw (m4.drain) -- (m6.drain);
		\draw (m5.gate) -- (m6.gate);
		\draw (m5.source) -- (m6.source);
		
		%stage2
		\path (2,0) coordinate(2nd-stage-offset);
		\draw (m6.drain) ++(2nd-stage-offset) node[nmos, anchor=gate](m7){};
		\draw (m7.base) node[right=20]{$M3$};
		\draw (m6.drain) node[circ]{} -- (m7.gate); %net3
%		\path ($(m6.drain)!0.2!(m4.source)$) node[right]{$net_{3}$};
		\draw (m7.source) |- (m6.source) node[circ]{}; %netGnd
%		\path ($(m5.source)!0.3!(m6.source)$) node[below]{$net_{Gnd}$};
		
		\draw (m7.drain) -- ++(0,.75) coordinate(outnode);
		\draw 
			let 
				\p1=(m2.drain),
				\p2=(outnode)
			in
				(\x2, \y1) node[pmos, anchor=drain](m8){};
		\draw (m8.drain) -- (outnode);
		\draw (m8.source) |- (m2.source) node[circ]{};
		\draw (m8.gate) -- (m2.gate);
		\draw (m8.base) node[right=20]{$M5$};
%		\draw (outnode)  to[R,l_=$R$] ++(0,3) |- (m2.source) node[circ]{};
		\path ($(m6.drain)!0.7!(m7.gate)$) coordinate(m6-m7-midnode);
		\draw (outnode) to[/tikz/circuitikz/bipoles/length=0.7cm, C,l_=$C_1$] ++(-1.,0) -| ($(m6-m7-midnode)+(.3,0)$) node[circ]{};
		\draw (outnode) to[short, *-o] ++(.5,0) node[below]{\small$V_{out}$};
%		\path ($(outnode)!0.5!(m8.drain)$) node[right]{$net_{out}$};
		
		%GND
		\draw (m6.S) node[ground]{};
%		\draw (m6.S) to[short,*-o] ++(0,-1) node[right]{$Gnd$};
		
	\end{circuitikz}
\end{document}
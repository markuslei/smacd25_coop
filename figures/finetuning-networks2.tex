 %\documentclass[tikz, border=20]{standalone}
 %\documentclass[tikz]{standalone}
 %\documentclass[border=1mm]{standalone}
 \documentclass[]{standalone}
 \usepackage{tikz}
 \usepackage[siunitx]{circuitikz} % Loading circuitikz with siunitx option
 \usetikzlibrary{calc}
 \usetikzlibrary{shapes.symbols}
 \usepackage{etoolbox}
 \usepackage{xcolor}
 \usepackage{bm}
 
 \makeatletter
 \tikzset{
 	database/.style={
 		path picture={
 			\draw (0, 1.5*\database@segmentheight) circle [x radius=\database@radius,y radius=\database@aspectratio*\database@radius];
 			\draw (-\database@radius, 0.5*\database@segmentheight) arc [start angle=180,end angle=360,x radius=\database@radius, y radius=\database@aspectratio*\database@radius];
 			\draw (-\database@radius,-0.5*\database@segmentheight) arc [start angle=180,end angle=360,x radius=\database@radius, y radius=\database@aspectratio*\database@radius];
 			\draw (-\database@radius,1.5*\database@segmentheight) -- ++(0,-3*\database@segmentheight) arc [start angle=180,end angle=360,x radius=\database@radius, y radius=\database@aspectratio*\database@radius] -- ++(0,3*\database@segmentheight);
 		},
 		minimum width=2*\database@radius + \pgflinewidth,
 		minimum height=3*\database@segmentheight + 2*\database@aspectratio*\database@radius + \pgflinewidth,
 	},
 	database segment height/.store in=\database@segmentheight,
 	database radius/.store in=\database@radius,
 	database aspect ratio/.store in=\database@aspectratio,
 	database segment height=0.1cm,
 	database radius=0.25cm,
 	database aspect ratio=0.35,
 }
 \makeatother
 
 \tikzset{%
 	every neuron/.style={
 		circle,
 		draw,
 		minimum size=.4cm
 	},
 	neuron 100/.style={
 		draw=none, 
 		scale=1.3,
 		text height=0.333cm,
 		execute at begin node=\color{black}$\vdots$
 	},
 }
 
 \tikzset{%
 	every neuron2/.style={
 		circle,
 		draw,
 		minimum size=.6cm
 	},
 	neuron2 100/.style={
 		draw=none, 
 		scale=1.5,
 		text height=0.233cm,
 		execute at begin node=\color{black}$\vdots$
 	},
 }
 
 \tikzset{%
 	every empty-neuron/.style={
 		minimum size=.6cm
 	}
 }
 
 
 \begin{document}
 	\begin{circuitikz}
 		\ctikzset{tripoles/mos style=arrows}
 		\ctikzset{transistors/arrow pos=end}
 		
 		\tikzstyle{every node}=[font={{\Large }}]
 		
 		\newcommand{\diodecon}[1]{\draw (#1.gate) node[circ]{} |-  (#1.drain) node[circ]{}}
 		
 		
 		\begin{scope}[shift={(4,-9.5)}]
 			\coordinate (start) at (0,0);
 				
			%scm3
			\draw (start) node[nmos, anchor=source, xscale=-1, color=red](m1){};
			\draw (start) ++(3,0) node[nmos, anchor=source, color=red](m6){};
			\diodecon{m1};
			\draw (m1.source) -- (m6.source);
			\draw (m1.gate) -- (m6.gate);
			\draw (m1.base) node[left=30, text=red]{$M1$};
			\draw (m6.S) node[above left, text=red]{$M6$};
			
			%dp
			\coordinate (P) at (1.5,-3.2);
			\draw (P) ++(-1.2,0) node[nmos, anchor=source, color=green](m9){};
			\draw (P) ++(1.2,0) node[nmos, anchor=source, xscale=-1, color=green](m10){};
			\draw (m9.source) |- (P);
			\draw (P) -| (m10.source);
			\draw (m10.G) ++(0,0) to[short, o-] (m10.G);
			\draw (m9.G) ++(0,0) to[short, o-] (m9.G);
			\draw (m10.G) ++(.75, 0)  node[](vinm){$V_{in-}$};
			\draw (m9.G) ++(-.75,0)  node[](vinp){$V_{in+}$};
			\draw (m10.base) node[left=30, text=green]{$M10$};
			\draw (m9.base) node[right=30, text=green]{$M9$};
			
			%ccm1
			\coordinate (ccm1s) at (0,-7.5);
			\draw (ccm1s) node[nmos, anchor=source, xscale=-1, color=blue](m2){};
			\draw (ccm1s) ++(5,0) node[nmos, anchor=source, color=blue](m8){};
			\draw (ccm1s) ++(0,1.5) node[nmos, anchor=source, xscale=-1, color=blue](m3){};
			\draw (ccm1s) ++(5,1.5) node[nmos, anchor=source, color=blue](m7){};
			\diodecon{m2};
			\diodecon{m3};
			\draw (m2.S) -- (m8.S);

			\draw (m2.G) -- (m8.G);
			\draw (m3.G) -- (m7.G);
			\draw (m2.base) node[left=30, text=blue]{$M2$};
			\draw (m3.base) node[left=30, text=blue]{$M3$};
			\draw (m8.base) node[right=30, text=blue]{$M8$};
			\draw (m7.base) node[right=30, text=blue]{$M7$};
 		\end{scope}
 		
 		\draw[] (6.5,-17.5) node[scale=3]{\color{black}$\vdots$};
 		\draw[] (16.,-17.58) node[scale=3]{\color{black}$\vdots$};
 		
 		\path (start) ++(1.7,-.3) node[]{$\text{scm}_{{\color{red}bb},1}$};
 		\path (m9.S) ++(1.5,-.3) node[]{$\text{dp}_{\color{green}tt}$};
 		\path (m8.S) ++(-0.6,-.3) node[]{$\text{ccm}_{\color{blue}llll}$};
	
 		
 		\node[anchor=west] at (14.4,-7) {$\bm{y} \hspace{1.2cm} \rightarrow \hspace{1.2cm} \bm{x}$ };
 		\node[anchor=west, align=left] at (10.3,-7.2) {\textbf{OpAmp} \\ \textbf{performance}};
 		\node[anchor=west, align=left] at (18.5,-7.2) {\textbf{device} \\ \textbf{size}};
 		
% 		\node[anchor=west, align=left] at (4,-7) {\textbf{training} \\ \textbf{samples}};
% 		\node[anchor=west, align=left] at (24,-7) {\textbf{test} \\ \textbf{samples}};
 		
 		\begin{scope}[shift={(2.5,-8)}]
 			\begin{scope}[shift={(13,-.5)}, scale=.6]
 				\foreach \m/\l [count=\y] in {1,2,100,3,4}
 				{
 					\node [every neuron/.try, neuron \m/.try] (input-\m) at (0,2.5-\y) {};
 					%		  		\ifnumequal{\m}{100}{}{\node[] at (-.7,2.5-\y) {{\normalsize $\phi_\m$}};}
 				}
 				\foreach \m/\l [count=\y] in {1,2,100,K-1,K}
 				{
 					\ifnumequal{\y}{3}{}{\node[anchor=west] at (-1.8,2.5-\y) {{\normalsize $y_{\m}$}};}
 				}
 				
 				\foreach \m [count=\y] in {1, 2, 100,3,4}
 				{
 					\node [every neuron/.try, neuron \m/.try ] (hidden-\m) at (2,2.5-\y) {};
 				}
 				
 				\foreach \m [count=\y] in {1,6}
 				{
 					\node [every neuron/.try, neuron \m/.try ] (output-\y) at (4,1-\y) {};
 					\node[] at (4.7,1-\y) {{\footnotesize $\frac{W_\m}{L_\m}$}};
 				}
 				
 				\foreach \i in {1,...,4}
 				\foreach \j in {1,...,4}
 				\draw [->] (input-\i) -- (hidden-\j);
 				
 				\foreach \i in {1,...,4}
 				\foreach \j in {1,...,2}
 				\draw [->] (hidden-\i) -- (output-\j);
 				
 				\draw (1,-2.8) node[below, text width=4.5cm, align=center]{scm$_{{\color{red}bb}}$};
 			\end{scope}
 			
 			
 		\end{scope}
 		\begin{scope}[shift={(2.5,-11)}]	
 			\draw [-latex, line width=3mm, align=center](8.5,-1.25) -- node[above]{fine-tuning \\ (transfer learning)} (10.5,-1.25);
 			\draw (9.6,-2.) node[align=center, font=\fontsize{9pt}{5pt}\selectfont]{of NNs from library in fig. 1\\ in context of the circuit in fig. 2a.};
% 			\draw [-latex, line width=3mm](16.5,-1.25) -- node[above]{evaluate} (18.5,-1.25);
 			\begin{scope}[shift={(13,-.9)}, scale=.6]
 				\foreach \m/\l [count=\y] in {1,2,100,3,4}
 				{
 					\node [every neuron/.try, neuron \m/.try] (input-\m) at (0,2.5-\y) {};
 					%		  		\ifnumequal{\m}{100}{}{\node[] at (-.7,2.5-\y) {{\normalsize $\phi_\m$}};}
 				}
 				\foreach \m/\l [count=\y] in {1,2,100,K-1,K}
 				{
 					\ifnumequal{\y}{3}{}{\node[anchor=west] at (-1.8,2.5-\y) {{\normalsize $y_{\m}$}};}
 				}
 				
 				\foreach \m [count=\y] in {1, 2, 100,3,4}
 				{
 					\node [every neuron/.try, neuron \m/.try ] (hidden-\m) at (2,2.5-\y) {};
 				}
 				
 				\foreach \m [count=\y] in {9,10}
 				{
 					\node [every neuron/.try, neuron \m/.try ] (output-\y) at (4,1-\y) {};
 					\node[] at (4.7,1-\y) {{\footnotesize $\frac{W_{\m}}{L_{\m}}$}};
 				}
 				
 				\foreach \i in {1,...,4}
 				\foreach \j in {1,...,4}
 				\draw [->] (input-\i) -- (hidden-\j);
 				
 				\foreach \i in {1,...,4}
 				\foreach \j in {1,...,2}
 				\draw [->] (hidden-\i) -- (output-\j);
 				
 				\draw (1,-2.5) node[below, text width=4.5cm, align=center]{dp$_{{\color{green}tt}}$};
 			\end{scope}
 		\end{scope}
 		
 		\begin{scope}[shift={(2.5,-14)}]
 			\begin{scope}[shift={(13,-1.2)}, scale=.6]
 				\foreach \m/\l [count=\y] in {1,2,100,3,4}
 				{
 					\node [every neuron/.try, neuron \m/.try] (input-\m) at (0,2.5-\y) {};
 					%		  		\ifnumequal{\m}{100}{}{\node[] at (-.7,2.5-\y) {{\normalsize $\phi_\m$}};}
 				}
 				\foreach \m/\l [count=\y] in {1,2,100,K-1,K}
 				{
 					\ifnumequal{\y}{3}{}{\node[anchor=west] at (-1.8,2.5-\y) {{\normalsize $y_{\m}$}};}
 				}
 				
 				\foreach \m [count=\y] in {1, 2, 100,3,4}
 				{
 					\node [every neuron/.try, neuron \m/.try ] (hidden-\m) at (2,2.5-\y) {};
 				}
 				
 				\foreach \m [count=\y] in {2,3,7,8}
 				{
 					\node [every neuron/.try, neuron \m/.try ] (output-\y) at (4,2-\y) {};
 					\node[] at (4.7,2-\y) {{\footnotesize $\frac{W_\m}{L_\m}$}};
 				}
 				
 				\foreach \i in {1,...,4}
 				\foreach \j in {1,...,4}
 				\draw [->] (input-\i) -- (hidden-\j);
 				
 				\foreach \i in {1,...,4}
 				\foreach \j in {1,...,4}
 				\draw [->] (hidden-\i) -- (output-\j);
 				
 				\draw (1,-2.8) node[below, text width=4.5cm, align=center]{scm$_{{\color{blue}ll}}$};
 			\end{scope}
 			
 		\end{scope}
 		
 		
 		
 	\end{circuitikz}
 \end{document}
